% pdflatex
\documentclass[12pt]{article}

%\usepackage[english]{babel} 
\usepackage[french]{babel}
\usepackage[utf8x]{inputenc}
\usepackage[OT1]{fontenc}
\usepackage{fullpage}
\usepackage{hyperref}
\usepackage{xspace}
\usepackage{pbox} % parbox of variable width
\usepackage{framed}
%%\usepackage{bibunits} %% separate biblio into several parts (section or chapter)

%% %\usepackage[timeinterval=10]{tdclock}
\usepackage{graphicx}
%% \usepackage{tikz}
%% %\usepackage{changepage}     % for the adjustwidth environment
%% \usepackage[tikz]{bclogo}   % nice boxes with logo
%% \usepackage{amsmath}
%% \usepackage{bm} %boldmath
%% \usepackage{listingsutf8}
%% \lstloadlanguages{R}


% ------- Notations -------------------------------------------------
%% \input{notations_u}
% -------------------------------------------------------------------
%%% Mettre en valeur
\newcommand{\insist}[1]{{\color{blue}\textbf{#1}}}
\newcommand{\warn}[1]{{\color{red}\textbf{#1}}}

% ****************************************************************************
% *************************************************************** Indian Names
\def\kihai{kihai}
\def\tala{tâla}
\def\theka{thekâ}
\def\tihai{tihai}


\def\jhaptal{jhaptâl}
\def\keherwa{keherwa}
\def\tintal{tintal}

%%% Matra from Math
\usepackage{amsmath,mathabx}
\newcommand{\matra}[1]{$\undergroup{\text{#1}}$}

%%% Gamak
\newcommand{\gamak}[1]{$\overrightarrow{\text{#1}}$}
%%% Press
%\newcommand{\press}[1]{$\breve{\text{#1}}$}
\newcommand{\press}[1]{$\overset{\bullet}{\text{#1}}$}
%%% Accent
\newcommand{\acc}[1]{$\underline{\text{#1}}$}

%% bool pour doigté
\newif\ifdoigt
%\doigttrue
\newcommand{\bol}[2]{%
  \ifdoigt
  \pbox[b]{2cm}
       {\hspace*{\fill}{\scriptsize #2}\\#1}
  \else
      {#1}
  \fi
}%
\def\Go{\bol{Ge}{1}}
\def\Gd{\bol{Ge}{2}}
\def\K{\bol{Ke}{}}
\def\Ka{\bol{Ka}{}}
\def\Ko{\bol{Ko}{}}
\def\Ki{\bol{Ki}{}}

\def\To{\bol{Te}{gn1}}
\def\Ro{\bol{Re}{gn1}}
\def\Tt{\bol{Te}{g3}}
\def\Ti{\bol{Ti}{g3}}
\def\ti{\bol{ti}{gn2}}
\def\Rt{\bol{Re}{g3}}
\def\N{\bol{Na}{k}}
\def\Ta{\bol{Ta}{k}}
%\def\Ti{\bol{Ti}{s/k}}
\def\Thi{\bol{Thin}{s/k}}
\def\Taa{\bol{Ta!}{g4}}
\def\Tin{\bol{Tin}{sn1}}
\def\Tu{\bol{Tun}{sn1}}

\def\Da{\bol{Dha}{k/2}}
\def\Dao{\bol{Dha}{k/1}}
\def\Di{\bol{Dhin}{gsn1/2}}
\def\Du{\bol{Dhun}{}}
\def\Ka{\bol{Ka}{}}

\def\sep{ / }
\def\sepnl{\\}

\def\cont{\bol{+}{}}

\newcommand{\double}[1]{%
  #1\bol{+}{}
  }%

\newcounter{nbexo}% Counter
\setcounter{nbexo}{1}
\newcommand{\exo}[1]{%
  %\begin{framed}
  \subsection*{Exercice \thenbexo{} #1}
  \stepcounter{nbexo}
  %\end{framed}
}%

\newcounter{nbtal}% Counter
\setcounter{nbtal}{1}
%% \newcommand{\kihai}[1]{%
%%   \begin{framed}
%%     \subsection*{Kihai \thenbtal{} #1}
%%     \stepcounter{nbtal}
%%   \end{framed}
%% }%
\newcommand{\subtitle}[1]{%
  \begin{framed}
    \subsection*{#1}
  \end{framed}
}%
  

\newcommand{\var}[1]{%
  \vspace{1em}
  \subsubsection*{Variante #1}
}

\def\GGTT{\matra{\Go \Gd \Tt \To}}
\def\KTKT{\matra{\Ka \Ta \Ka \Ta}}
\def\TRKT{\matra{\Tt \Ro \K \Tt}}
\def\KTTK{\matra{\K \Tt \To \K}} 

\setlength{\parindent}{0cm} % Default is 15pt.
%\linespread{1.3}

% Document title, author, date and institution
% ---------------------------------------------
\title{%
  Tabla
}
%\subtitle
% ****** Modif si Date différent de Date du jour
%\date{04/10/2017}

% =============================================================================
\begin{document}
\bibliographystyle{apalike}

\maketitle

\doigttrue
\exo{}
\Da \Tt \To \hspace{1cm} et \Da \To \Tt
%\vspace{1em}

\exo{}
\Gd \Go \Gd \sep \Go \Gd \Go \sep \K \K \K \sep \Go \Gd \Go
\vspace{1em}

\Da \Go \Gd \sep \Dao \Gd \Go \sep \Ka \K \K \sep \Da \Gd \Go


\exo{(TeReKeTe - KeTeTaKa)}
\Tt \To \K \Tt \sep \K \Tt \To \K

\exo{(TeReKeTeTaKa)}
\Tt \Ro \K \Tt \Ta \Ka

\exo{Shuffle}

\Da \Go \N \Ti \sep \N \K \Di \N \sepnl

Avec accent sur 1 (\Da) et 3 (\N), puis shuffle pour le groove.

\exo{GKKG - brazil}

\K \Tt \N \K \sep \Di \To \N \Gd
\hspace{1cm}ou\hspace{1cm} 
\K \Tt \N \K \sep \Gd \To \N \Gd

% ****************************************************************************
% ***************************************************************** Practice 1 
% ****************************************************************************
\newpage
%\kihai{Practice 1}
\subtitle{Les Grandes Practices}
\doigtfalse
\subsubsection*{Practice 1}
(\matra{\double{\Da} \Tt \Ro} \matra{\K \Tt}) x 4 \sepnl
(\matra{\double{\Da} \Tt \Ro} \matra{\K \Tt \Ta \Ka}) x 4 \sepnl
\matra{\double{\Tu} \double{\N}} \matra{\K \Tt \Ta \Ka} %\sepnl
%% + Moudi
%% \vspace{1em}

%% Ce qui donne:
%% \vspace{1em}

%% \matra{\double{\Da} \Tt \Ro} \matra{\K \Tt \double{\Da}} \sep
%% \matra{\Tt \Ro \K \Tt} \matra{\double{\Da} \Tt \Ro} \sepnl
%% \matra{\K \Tt \double{\Da}} \matra{\Tt \Ro \K \Tt} \sep
%% \matra{\double{\Da} \Tt \Ro} \matra{\K \Tt \Ta \Ka} \sepnl
%% \vspace{1em}

%% \matra{\double{\Da} \Tt \Ro} \matra{\K \Tt \Ta \Ka} \sep
%% \matra{\double{\Da} \Tt \Ro} \matra{\K \Tt \Ta \Ka} \sepnl
%% \matra{\double{\Da} \Tt \Ro} \matra{\K \Tt \Ta \Ka} \sep
%% \matra{\double{\Tu} \double{\N}} \matra{\K \Tt \Ta \Ka} \sepnl
%% \vspace{1em}

%% \matra{\double{\Ta} \Tt \Ro} \matra{\K \Tt \double{\Ta}} \sep
%% \matra{\Tt \Ro \K \Tt} \matra{\double{\Ta} \Tt \Ro} \sepnl
%% \matra{\K \Tt \double{\Ta}} \matra{\Tt \Ro \K \Tt} \sep
%% \matra{\double{\Ta} \Tt \Ro} \matra{\K \Tt \Ta \Ka} \sepnl
%% \vspace{1em}

%% \matra{\double{\Da} \Tt \Ro} \matra{\K \Tt \Ta \Ka} \sep
%% \matra{\double{\Da} \Tt \Ro} \matra{\K \Tt \Ta \Ka} \sepnl
%% \matra{\double{\Da} \Tt \Ro} \matra{\K \Tt \Ta \Ka} \sep
%% \matra{\double{\Tu} \double{\N}} \matra{\K \Tt \Ta \Ka} \sepnl
%% \vspace{1em}


% ****************************************************************************
% ***************************************************************** Practice 2 
% ****************************************************************************
%\newpage
%\kihai{Practice 2}
\subsubsection*{Practice 2}
(\matra{\double{\Da}  \Tt \Ro} \matra{\K \Tt \Ta \Ka} \matra{\Tt \Ro \K \Tt}) x 2 \sepnl
%\matra{\double{\Da}  \Tt \Ro} \matra{\K \Tt \Ta \Ka} \matra{\Tt \Ro \K \Tt} \sepnl
\matra{\double{\Da} \Tt \Ro} \matra{\K \Tt \Ta \Ka} %\sepnl
%% x 2 + Moudi
%% \vspace{1em}

%% Ce qui donne:
%% \vspace{1em}

%% \matra{\double{\Da}  \Tt \Ro} \matra{\K \Tt \Ta \Ka} \sep
%% \matra{\Tt \Ro \K \Tt} \matra{\double{\Da}  \Tt \Ro} \sepnl
%% \matra{\K \Tt \Ta \Ka} \matra{\Tt \Ro \K \Tt} \sep
%% \matra{\double{\Da} \Tt \Ro} \matra{\K \Tt \Ta \Ka} \sepnl
%% \vspace{1em}

%% \matra{\double{\Da}  \Tt \Ro} \matra{\K \Tt \Ta \Ka} \sep
%% \matra{\Tt \Ro \K \Tt} \matra{\double{\Da}  \Tt \Ro} \sepnl
%% \matra{\K \Tt \Ta \Ka} \matra{\Tt \Ro \K \Tt} \sep
%% \matra{\double{\Da} \Tt \Ro} \matra{\K \Tt \Ta \Ka} \sepnl
%% \vspace{1em}

%% \matra{\double{\Ta}  \Tt \Ro} \matra{\K \Tt \Ta \Ka} \sep
%% \matra{\Tt \Ro \K \Tt} \matra{\double{\Ta}  \Tt \Ro} \sepnl
%% \matra{\K \Tt \Ta \Ka} \matra{\Tt \Ro \K \Tt} \sep
%% \matra{\double{\Ta} \Tt \Ro} \matra{\K \Tt \Ta \Ka} \sepnl
%% \vspace{1em}

%% \matra{\double{\Da}  \Tt \Ro} \matra{\K \Tt \Ta \Ka} \sep
%% \matra{\Tt \Ro \K \Tt} \matra{\double{\Da}  \Tt \Ro} \sepnl
%% \matra{\K \Tt \Ta \Ka} \matra{\Tt \Ro \K \Tt} \sep
%% \matra{\double{\Da} \Tt \Ro} \matra{\K \Tt \Ta \Ka}

%% \newpage
%% \doigtfalse
%% \kihai{Tintal}
%% \Gd \Go \Tt \To \sep \Gd \Go \N \N \sep \K \K \Tt \To \sep \Gd \Go \N \N

\subsubsection*{Majax}
\matra{\double{\Da} \Tt \Ro} \matra{\K \Tt \Ta \Ka} \sep
\matra{\Taa \Ka \Tt \Ro}    \matra{\K \Tt \Ta \Ka} \sepnl
%%\vspace{1em}
\matra{\Tt \To \Tt \To}  \matra{\K \Tt \Ta \Ka} \sep
\matra{\double{\Tu} \double{\N}} \matra{\K \Tt \Ta \Ka} 

\subsubsection*{Practice 3 : Nabanku}
\matra{\double{\Da}\double{\Da}} \matra{\double{\Da}\Tt \Ro}
\matra{\K \Tt \Ta \Ka} \matra{\Taa \Ka \Tt \Ro} \sepnl
\matra{\K \Tt \Ta \Ka} \matra{\double{\Da} \Tt \Ro}
\matra{\K \Tt \Ta \Ka} \matra{\double{\Da}\double{\Da}} + \textbf{Majax}

\subsubsection*{Practice 4}
(\matra{\double{\Da} \double{\ti}} \matra{\double{\Da} \Tt \Ro}
\matra{\K \Tt \Ta \Ka}  \TRKT ) x 2 \sepnl
\matra{\double{\Da} \Tt \Ro} \matra{\K \Tt \N \Ka}
\TRKT \matra{\N \Ka \N \Ka} \sepnl
\TRKT \matra{\Ta \Ka \Tt \Ro} \matra{\K \Tt \Ta \Ka} \TRKT 

\subsubsection*{Practice 5}
\matra{\double{\Da} \K \Rt} \matra{\N \Ka \Tt \Ro}
\matra{\K \Tt \To \Go} \matra{\double{\Da} \Tt \Ro} \sepnl
\matra{\K \Tt \Ta \Ka} \matra{\Tt \Ro \K \Tt}
\matra{\Ta \Ka \Taa \Ka} \TRKT \sepnl
\matra{\double{\Da} \Tt \Ro} \matra{\K \Tt \Ta \Ka}
\matra{\Tt \Ro \K \Tt} \matra{\Ta \Ka \double{\Da}} \sepnl
\matra{\double{\Da} \Tt \Ro} \matra{\K \Tt \Ta \Ka}
\matra{\Taa \Ka \Tt \Ro} \matra{\K \Tt \Ta \Ka}

\subsection*{Practice Debo Nov17}
\matra{\double{\Da} \Tt \Ro} \matra{\K \Tt \double{\Da}}
\matra{\Tt \Ro \K \Tt} \matra{\Ta \Ka \Tt \Ro} \sepnl
\matra{\K \Tt \Ta \Ka} \matra{\Taa \Ka \Tt \Ro}
\matra{\K \Tt \double{\Da}} \matra{\Tt \Ro \K \Tt}

\subsection*{Practice Debo Juin23}
\matra{\double{\Da}\Tt\Ro} \matra{\K\Tt\double{\Da}} \TRKT \matra{\Tt\K\double{\Da}} \sepnl
\TRKT \matra{\double{\Da}\Tt\Ro} \KTTK \TRKT

\subsection*{Practice Debo Juin23}

\matra{\Da\To\Tt} \matra{\Da\ti} \matra{\Da\To\Tt} \sepnl
\matra{\Ta\To\Tt} \matra{\Da\ti} \matra{\Da\To\Tt} \sepnl

Et remplacer \To\Tt par \To\Tt\To\Tt ou \Tt \Ro \K \Tt ou \Gd\Go\Gd\Go.

Peut aussi remplacer \matra{\Da\To\Tt} par \matra{\Da\Tu\N}.

\subsection*{\Da\ti}

Les accents, un peu comme à la batterie, en alternant temps et en l'air.

\acc{\Da}\ti \acc{\Da}\ti \sep \acc{\Da} \ti \acc{\Da}\Da \sep

\acc{\ti} \Da \acc{\ti} \Da \sep \acc{\ti} \Da \acc{\ti} \Da \sep


% ****************************************************************************
% ******************************************************* Workshop Debo juin23
% ****************************************************************************
\newpage
\subtitle{Workshop Debo juin23}

Sur une base commune, on ajoute au milieu des Tekka de différentes métrique. Tout fonctionne en Tintal. Sachant qu'on peut aussi doubler des phrases ou des matra en les jouant
deux fois plus vite...

\vspace{3mm}
\textbf{Roulement}

\matra{\Go\N} \matra{\Da\ti} \matra{\press{\Da}\Da} \matra{\Go\N} \sepnl
\matra{\double{\Da}} \matra{\double{\Da}\Tt\Ro} \KTTK \TRKT x2 \sepnl
\matra{\Da\ti} \matra{\Da\Go} \matra{\Tu\N} \matra{\K\N}

\vspace{3mm}
\textbf{Japtap}

\matra{\Go\N} \matra{\Da\ti} \matra{\double{\Da}} \sepnl
\matra{\Di\N} \matra{\press{\Di}\Di} \matra{\N\Thi} \matra{\N\press{\Di}} \matra{\Di\N} \sepnl
\matra{\double{\Da}} \matra{\double{\Da}\Tt\Ro} \KTTK \TRKT \sepnl
\matra{\Da\ti} \matra{\Da\Go} \matra{\Tu\N} \matra{\K\N}

\vspace{3mm}
\textbf{Rupak}

\matra{\Go\N} \matra{\Da\ti} \matra{\double{\Da}} \sepnl
\matra{\cont\Di} \matra{\Di\N} \matra{\Di\N} \matra{\Di\N} \matra{\double{\Da}} \sepnl
\matra{\cont\Thi} \matra{\Thi\N} \matra{\Di\N} \matra{\Di\N} \sepnl
\matra{\Da\ti} \matra{\Da\Go} \matra{\Tu\N} \matra{\K\N}
% ****************************************************************************
% *************************************************************** Kaida JMLP-2
% ****************************************************************************
\newpage
\subtitle{Kaïda JMLP-2 - \tintal{}}

\begin{tabular}{lllll}
  A & \Da \Go & \Ti \To & \Gd \Go & \Tt \To \\
  &\Gd \Go &   \Ti \To & \Ki \To & \Ta \Ka \\
  \hline
  B & \Ta \acc{\Go} & \Ti \To & \K \K & \Tt \To \\
  & \Gd \Go &   \Ti \To & \Ki \To & \Ta \Ka \\
\end{tabular}

  \textbf{var 1} \\
  (\Da \Go TT \sep GGTT) x 2 + A \hspace{1cm}
  (\Ta \Go TT \sep KKTT) x 2 + A

  \textbf{var 2} \\
  \Da \Go TT \sep (GGTT) x 3 + A \hspace{1cm}
  \Ta \Go TT \sep (KKTT) x 3 + A

  \textbf{var 3} \\
  \Da \Go TT \sep \Go{}TT\Gd \sep \Go\Gd{}TT \sep KTTK + A\\
  \Ta \Go TT \sep KTTK \sep KKTT \sep KTTK + A

  \textbf{var 4} \\
  \Da \Go TT \sep TTGG \sep GGTT \sep KTTK + A\\
  \Ta \Go TT \sep TTKK \sep KKTT \KTTK + A


% ****************************************************************************
% ***************************************************************** Kihai Amit
% ****************************************************************************
\newpage
%\doigttrue
%\kihai{Tintal - Maître Amit}
\subtitle{\kihai{} «Maître Amit» - \tintal{}}
\GGTT{}                 \GGTT{} \sep
\matra{\Go \Gd \N \Go}  \matra{ \Di \N \Go \N} \sepnl
\vspace{1em}

\KTKT{} \GGTT{} \sep
\matra{\Go \Gd \N \Go}  \matra{ \Di \N \Go \N}
\doigtfalse

\var{1}
\GGTT{}                 \matra{\Go \Gd \Go \Gd} \sep
\matra{\Tt \To \Go \Gd} \GGTT{} \sepnl
\KTKT{}                 \GGTT{} \sep
\matra{\Go \Gd \N \Go}  \matra{\Di \N \Go \N}

+ thème + Moudi + thème

\var{2}
(\GGTT{} \matra{\Go \Gd \N \Go} \Di \N ) x 2 \sepnl
\GGTT{} \matra{\Go \Gd \N \Go}  \matra{ \Di \N \Go \N}

+ thème + Moudi + thème

\var{3}
(\matra{\Go \Gd \N \Go}  \matra{ \Di \N \Go \N}) x 4

+ thème + Moudi + thème

\var{4}
\GGTT{} ( \matra{\Go \Gd \Go \Gd} \Tt \To) x 4 \sepnl
\GGTT{}

+ thème + Moudi + thème

\subsubsection*{Tihaï}

\GGTT{} \GGTT{} \sep
\matra{\Go \Gd \N \Go}  \matra{\Di \N \Go \N} \sepnl

(\double{\Da} \matra{\Gd \Go \N \Go} \matra{\Di \N \Go \N} ) x 2

\matra{\double{\Da} \K \Ta} \matra{\double{\underline{\Da}} \double{\cont}} (dernier ``one'' seulement pour 1 et 2)




\newpage
%\kihai{Laggi (p191)}
\subtitle{Laggi (p191)}
Accompagnement léger, par exemple d'un Bajan comme ce qui suit

\matra{\double{\Di} \N \Di} \matra{\cont \Di \N \To} \sep
\matra{\double{\Tin} \N \Tin} \matra{\cont \Tin \N \K} \sepnl

\vspace{1em}
Le Laggi, en mnémotechnique (``Kinar must be \emph{bright}'')

\vspace{0.5em}
\Da \Go \N \N \Go \N \sep 
( \Da \Go \N \Go \N ) x 2 \sepnl
%\Da \Go \N \Gd \N \sepnl


%Ce qui donne ensuite, détaillé, le moudi conserve la structure
\doigtfalse

\vspace{1em}
\matra{\Da \Go \N \N} \sep \matra{\Go \N \Da \Go} \sep
\matra{\N \Gd \N \Da} \sep \matra{\Go \N \Gd \N} \sepnl

\matra{\Ta \K \N \N} \sep \matra{\K \N \Da \Go} \sep
\matra{\N \Gd \N \Da} \sep \matra{\Go \N \Gd \N} \sepnl

\paragraph{Var 2}
\Da \Go \N \N \Go \N \sep \Da \Go \N \N \Go \N \sep
\Da \Go \N \N \sep

\paragraph{Var 3}
\Da \Go \N \N \Go \N \sep \Da \Go \N \N \sep \Da \Go \N \N \Go \N \sep

\paragraph{Var 4}
\Da \Go \N \N \Go \N \sep \cont \Da \Go \N \Da \Go \N \Da \Go \N \sep

\paragraph{Var 5}
\Da \Go \N \N \Go \N \sep \cont \Da \Go \N \sep \cont \Da \Go \N \Go \N \sep

\paragraph{Var 6}
\Da \Go \N \N \sep \Da \Go \N \N \sep \Da \Go \N \N \sep \Gd \N  \Go \N \sep

\paragraph{Var 7}
\Da \Go \N \N \sep \cont \Da \Go \N \N \sep \cont \Da \Go \N \N \Go \N \sep

\paragraph{Var 8}
\Da \Go \N \N \sep \Go\N\Go\N\sep \Go\N\Go\N\sep \Go\N\Go\N\sep

\paragraph{Var 9}
\Da \Go \N \N \Go \N \sep \cont \Go\N\sep \cont \Go\N\sep \cont \Go\N\N\sep

\paragraph{Var 10}
\Da \Go \N \N \Go \N \sep \cont \Go\N\Go\N\Go\N\sep \Da \Go \N

\paragraph{Var 11}
\Da \Go \N \N \Go \N \sep \cont \Da \N\N\N \sep \Go\N \sep \Da\Go\N

\subsubsection*{Tihaï}
\Da \Go \N \N \Go \N \sepnl
( \cont \Da\N\N \sep \Go\N \Da \Go\N\N\N \cont \sep \cont \cont) x3 
%\cont \Go\N\N \sep \Go\N \Da \Go\N\N\N \cont \sep
%\cont \Go\N\N \sep \Go\N \Da \Go\N\N\N \cont \sep \cont \cont

% ****************************************************************************
% *************************************************************** Tintal Tekka 
% ****************************************************************************
\section{Tekka en Tintal}

\Da \Di \Di \Da \sep \Da \Di \Di \Da \sepnl
\Da \Thi \Thi \Ta \sep \Ta \Di \Di \Da \sepnl

Une version en [ press Dha gamak Dhin open Dhin Dha]

Exercice : remplacer un matra par
\begin{itemize}
\item binaire \matra{\Ta\Ta}
\item ternaire \matra{\Ta\K\N} ou \matra{\Da\Go\N} \matra{\N\Go\N}
\item quaternaire \TRKT ou \TRKT\KTTK ou \matra{\cont \Tt\Ro} \matra{\K\Tt\Ta\Ka} \TRKT
\end{itemize}

\paragraph{Mukhra 1} A partir du 9° bol. (TODO)

\matra{\Da} \matra{\Tt\To} \matra{\Thi} \matra{\Ta} \sep 
\matra{\Ka} \matra{\Tt\cont \Tt\Ro} \matra{\K\Tt\Ta\Ka} \TRKT

\paragraph{Mukra 2} (TODO)\\

\matra{\Ta \ti} \matra{\Ta \K} \matra{\Tin \N} \matra{\K\N} \sep
\matra{\Da\ti} \matra{\Da\cont \Tt\Ro} \matra{\K\Tt\Ta\Ka} \TRKT




\newpage
% ****************************************************************************
% *********************************************************** Damien en Tintak 
% ****************************************************************************
%\kihai
\subtitle{Damien en Tintal}
\doigttrue
Le tekka :

\matra{\Da} \matra{\Di} \matra{\Di} \matra{\Da} \sepnl
\matra{\Da (\Da)} \matra{\Di} \matra{\Di} (\matra{\Da}  ou \matra{\Ta \K \N})\sepnl
\matra{\Da} \matra{\Tin} \matra{\Tin} \matra{\Ta} \sepnl
\matra{\Ta} \matra{\Di} \matra{\Di} \matra{\Da} \sepnl

Pour annoncer un break ternaire, le 8è temps peut être remplacé par une formule ternaire \matra{\Ta \K \N}.

Break à partir de 9è bol (TODO Vérifier que Ti est Tin ou Thin ou ti).

\matra{\Ta \K} \matra{\Tt \Ta} \matra{\K \Rt \N \Ka} \TRKT{} \sepnl
\matra{TI \Gd \double{\Da}} \TRKT{} \matra{TI \Gd \double{\Da}} \TRKT{} \sepnl\sepnl

Break en ternaire

\matra{\Da \Tt \To} \matra{\Da \Go \N} \matra{\Da \Go \Tu} \matra{\N \K \N}


% ****************************************************************************
% ******************************************* Accompagnement "Bombay JAYASHRI" 
% ****************************************************************************
%\kihai
\subtitle{Bombay AYASHRI - Shravanam}

En Dadra (6)

\Da \Di \N \sep \Ta \Tin \N \sepnl

avec petite variation

\Da \Di \N \sep \Ta \TRKT \sepnl


\Da \matra{\Tt \To} \matra{\Tt \To}  \sep \Ta \Di \Di \sepnl

\doigtfalse
En fait, des ``press'' et des ``gamak''
\press{\Da} \matra{\Tt \To} \matra{\Tt \To}  \sep \Ta \gamak{\Di} \gamak{\Di} \sepnl

Peut aussi inclure ce break (Practice 2) sut 3 temps

\matra{\double{\Da}  \Tt \Ro} \matra{\K \Tt \Ta \Ka} \matra{\Tt \Ro \K \Tt} \sepnl

% ****************************************************************************
% ********************************************************************* Japtal
% ****************************************************************************
\newpage
%% \begin{framed}
%% \section*{tâla «Jhaptâl» (10 temps)}
%% \end{framed}
\subtitle{\tala{} «\jhaptal{}» (10 temps)}


%\subsection*{Thekâ du \textbf{Jhaptâl}}
\subsection*{\theka{} du \textbf{\jhaptal{}}}

\begin{tabular}{lllll}
\Di & \N & \press{\Di} & \Di &  \N \\ \Thi & \N & \press{\Di} & \Di & \N \\
\hline

\textbf{var 1}\\

\Di & \matra{\N\N} & \press{\Di} & \Di & \matra{\N\N} \\
\Thi & \matra{\N\N} & \press{\Di} & \Di & \matra{\N\N}\\
\hline

\textbf{var 2} \\

\Di & \matra{\N\N} & \matra{\press{\Di}\Di} & \matra{\N\press{\Di}} & \matra{\Di\N} \\ \Thi & \matra{\N\N} & \matra{\press{\Di}\Di} & \matra{\N\press{\Di}} &\matra{\Di\N} \\
\hline

\textbf{var 3}\\

\Di & \matra{\N\N} & \matra{\Di\cont\cont\N} & \matra{\press{\Di}\Di} & \N \\ \Thi & \matra{\N\N} & \matra{\Di\cont\cont\N} & \matra{\press{\Di}\Di} & \N \\
\end{tabular}

\textbf{\kihai{}}

\matra{\Di} \matra{\N\Da} \matra{\K\N\Ta\Go} \matra{\Di\N} \sepnl
\matra{\cont \Da} \matra{\K\N\Ta\Go} \matra{\Di\N} \sepnl
\matra{\cont \Da} \matra{\K\N\Ta\Go} \matra{\Di\N}

\subsection*{\kihai{}}

\begin{tabular}{lllll}
A & B & C & D & E\\  
\matra{\Ko \Ta \Da \ti} & \matra{\press{\Da} \Go \N \N} & \matra{\ti \press{\Da} \Go \N} & \matra{\Da \ti \press{\Da} \Go} & \matra{\Tu \N \K \N}\\
\matra{\Ko \Ta \Ta \ti} & \matra{\Ta \K \N \N} & \matra{\ti \press{\Da} \Go \N} & \matra{\Da \ti \press{\Da} \press{\Go}} & \matra{\Di \N \Go \N} 
\end{tabular}

\textbf{var 1}
A + A + A + B + C

\textbf{var 2}
A + B + A + B + C

\textbf{var 3}
A + B + B + B + C

% ****************************************************************************
% ********************************************************************* Kherva
% ****************************************************************************
\newpage
\subtitle{\tala{} «\keherwa{}» (8 temps)}

\subsection*{\theka{}}

\begin{tabular}{llll}
  \press{\Da} & \Go & \N & \Thi \\
  \N & \K & \Di & \N
\end{tabular}

% ****************************************************************************
% ********************************************************************* Kherva
% ****************************************************************************
\subtitle{Groove ``Koucherov''}

\matra{\Da \cont \cont \K} \matra{\N\Gd\press{\Go}\cont}

En fait, d'autres patterns main droite sont possibles

\matra{\N \cont\cont\cont} \matra{\N \cont\cont\cont} \sepnl
\matra{\cont\cont \N\N} \matra{\cont\cont \N\N} \sepnl
\matra{\N \cont \N\N} \matra{\N \cont \N\N} \sepnl

Avec des variations et breaks

\matra{\Ta \K \N \Ta} \matra{\K \N \Ta \cont}

\matra{\Tu \Tt \N \Tu} \matra{\Tt \N \Tu \tt}

\matra{\Ta \Ta TRKT} \matra{\Ta \Ta TRKT} 



% ****************************************************************************
% ***************************************************************** Références 
% ****************************************************************************
%\section{Références}

\end{document}
