% pdflatex
\documentclass[12pt]{article}

%\usepackage[english]{babel} 
\usepackage[french]{babel}
\usepackage[utf8x]{inputenc}
\usepackage[OT1]{fontenc}
\usepackage{fullpage}
\usepackage{hyperref}
\usepackage{xspace}
\usepackage{pbox} % parbox of variable width
\usepackage{framed}
%%\usepackage{bibunits} %% separate biblio into several parts (section or chapter)

%% %\usepackage[timeinterval=10]{tdclock}
\usepackage{graphicx}
%% \usepackage{tikz}
%% %\usepackage{changepage}     % for the adjustwidth environment
%% \usepackage[tikz]{bclogo}   % nice boxes with logo
%% \usepackage{amsmath}
%% \usepackage{bm} %boldmath
%% \usepackage{listingsutf8}
%% \lstloadlanguages{R}


% ------- Notations -------------------------------------------------
%% \input{notations_u}
% -------------------------------------------------------------------
%%% Mettre en valeur
\newcommand{\insist}[1]{{\color{blue}\textbf{#1}}}
\newcommand{\warn}[1]{{\color{red}\textbf{#1}}}

%%% Matra from Math
\usepackage{amsmath,mathabx}
\newcommand{\matra}[1]{$\undergroup{\text{#1}}$}

%% bool pour doigté
\newif\ifdoigt
%\doigttrue
\newcommand{\bol}[2]{%
  \ifdoigt
  \pbox[b]{2cm}
       {\hspace*{\fill}{\scriptsize #2}\\#1}
  \else
      {#1}
  \fi
}%
\def\Go{\bol{Ge}{1}}
\def\Gd{\bol{Ge}{2}}
\def\K{\bol{Ke}{}}
\def\Ka{\bol{Ka}{}}

\def\To{\bol{Te}{gn1}}
\def\Ro{\bol{Re}{gn1}}
\def\Tt{\bol{Te}{g3}}
\def\Rt{\bol{Re}{g3}}
\def\N{\bol{Na}{k}}
\def\Ta{\bol{Ta}{}}
\def\Taa{\bol{Ta!}{g4}}
\def\Ti{\bol{Tin}{sn1}}
\def\Tu{\bol{Tun}{}}

\def\Da{\bol{Dha}{k/2}}
\def\Dao{\bol{Dha}{k/1}}
\def\Di{\bol{Dhin}{gsn1/2}}
\def\Du{\bol{Dhun}{}}
\def\Ka{\bol{Ka}{}}

\def\sep{ / }
\def\sepnl{\\}

\def\cont{\bol{+}{}}

\newcommand{\double}[1]{%
  #1\bol{+}{}
  }%

\newcounter{nbexo}% Counter
\setcounter{nbexo}{1}
\newcommand{\exo}[1]{%
  %\begin{framed}
  \subsection*{Exercice \thenbexo{} #1}
  \stepcounter{nbexo}
  %\end{framed}
}%

\newcounter{nbtal}% Counter
\setcounter{nbtal}{1}
\newcommand{\kihai}[1]{%
  \begin{framed}
    \subsection*{Kihai \thenbtal{} #1}
    \stepcounter{nbtal}
  \end{framed}
}%

\newcommand{\var}[1]{%
  \vspace{1em}
  \subsubsection*{Variante #1}
}

\def\GGTT{\matra{\Go \Gd \Tt \To}}
\def\KTKT{\matra{\Ka \Ta \Ka \Ta}}
\setlength{\parindent}{0cm} % Default is 15pt.
%\linespread{1.3}

% Document title, author, date and institution
% ---------------------------------------------
\title{%
  Tabla
}
%\subtitle
% ****** Modif si Date différent de Date du jour
%\date{04/10/2017}


% =============================================================================
\begin{document}
\bibliographystyle{apalike}

\maketitle

\doigttrue
\exo{}
\Da \Tt \To \hspace{1cm} et \Da \To \Tt
%\vspace{1em}

\exo{}
\Gd \Go \Gd \sep \Go \Gd \Go \sep \K \K \K \sep \Go \Gd \Go
\vspace{1em}

\Da \Go \Gd \sep \Dao \Gd \Go \sep \Ka \K \K \sep \Da \Gd \Go


\exo{(TeReKeTe - KeTeTaKa)}
\Tt \To \K \Tt \sep \K \Tt \To \K

\exo{(TeReKeTeTaKa)}
\Tt \Ro \K \Tt \Ta \Ka


% ****************************************************************************
% ***************************************************************** Practice 1 
% ****************************************************************************
\newpage
\kihai{Practice 1}
(\matra{\double{\Da} \Tt \Ro} \matra{\K \Tt}) x 4 \sepnl
(\matra{\double{\Da} \Tt \Ro} \matra{\K \Tt \Ta \Ka}) x 4 \sepnl
\matra{\double{\Tu} \double{\N}} \matra{\K \Tt \Ta \Ka} \sepnl
+ Moudi
\vspace{1em}

Ce qui donne:
\vspace{1em}

\matra{\double{\Da} \Tt \Ro} \matra{\K \Tt \double{\Da}} \sep
\matra{\Tt \Ro \K \Tt} \matra{\double{\Da} \Tt \Ro} \sepnl
\matra{\K \Tt \double{\Da}} \matra{\Tt \Ro \K \Tt} \sep
\matra{\double{\Da} \Tt \Ro} \matra{\K \Tt \Ta \Ka} \sepnl
\vspace{1em}

\matra{\double{\Da} \Tt \Ro} \matra{\K \Tt \Ta \Ka} \sep
\matra{\double{\Da} \Tt \Ro} \matra{\K \Tt \Ta \Ka} \sepnl
\matra{\double{\Da} \Tt \Ro} \matra{\K \Tt \Ta \Ka} \sep
\matra{\double{\Tu} \double{\N}} \matra{\K \Tt \Ta \Ka} \sepnl
\vspace{1em}

\matra{\double{\Ta} \Tt \Ro} \matra{\K \Tt \double{\Ta}} \sep
\matra{\Tt \Ro \K \Tt} \matra{\double{\Ta} \Tt \Ro} \sepnl
\matra{\K \Tt \double{\Ta}} \matra{\Tt \Ro \K \Tt} \sep
\matra{\double{\Ta} \Tt \Ro} \matra{\K \Tt \Ta \Ka} \sepnl
\vspace{1em}

\matra{\double{\Da} \Tt \Ro} \matra{\K \Tt \Ta \Ka} \sep
\matra{\double{\Da} \Tt \Ro} \matra{\K \Tt \Ta \Ka} \sepnl
\matra{\double{\Da} \Tt \Ro} \matra{\K \Tt \Ta \Ka} \sep
\matra{\double{\Tu} \double{\N}} \matra{\K \Tt \Ta \Ka} \sepnl
\vspace{1em}


\newpage
\doigtfalse
\kihai{Tintal}
\Gd \Go \Tt \To \sep \Gd \Go \N \N \sep \K \K \Tt \To \sep \Gd \Go \N \N

\doigttrue
\kihai{Tintal - Maître Amit}
\GGTT{}                 \GGTT{} \sep
\matra{\Go \Gd \N \Go}  \matra{ \Di \N \Go \N} \sepnl
\vspace{1em}

\KTKT{} \GGTT{} \sep
\matra{\Go \Gd \N \Go}  \matra{ \Di \N \Go \N}
\doigtfalse

\var{1}
\GGTT{}                 \matra{\Go \Gd \Go \Gd} \sep
\matra{\Tt \To \Go \Gd} \GGTT{} \sepnl
\KTKT{}                 \GGTT{} \sep
\matra{\Go \Gd \N \Go}  \matra{\Di \N \Go \N}

+ thème + Moudi + thème

\subsubsection*{Tihaï}

\GGTT{} \GGTT{} \sep
\matra{\Go \Gd \N \Go}  \matra{\Di \N \Go \N} \sepnl

(\double{\Da} \matra{\Gd \Go \N \Go} \matra{\Di \N \Go \N} ) x 2

\matra{\double{\Da} \K \Ta} \matra{\double{\underline{\Da}} \double{\cont}} (dernier ``one'' seulement pour 1 et 2)



\newpage
\kihai{Tintal - Majax}
\doigttrue
\matra{\double{\Da} \Tt \Ro} \matra{\K \Tt \Ta \Ka} \sep
\matra{\Taa \Ka \Tt \Ro}    \matra{\K \Tt \Ta \Ka} \sepnl
\vspace{1em}

\matra{\Tt \To \Tt \To}  \matra{\K \Tt \Ta \Ka} \sep
\matra{\double{\Tu} \double{\N}} \matra{\K \Tt \Ta \Ka} 

\newpage
\kihai{Laggi}
Accompagnement léger, par exemple d'un Bajan comme ce qui suit

\matra{\double{\Di} \N \Di} \matra{\cont \Di \N \To} \sepnl
\matra{\double{\Ti} \N \Ti} \matra{\cont \Ti \N \K} \sepnl

\vspace{1em}
Le Laggi, en mnémotechnique

\vspace{1em}
\Da \Go \N \N \Go \N \sepnl
\Da \Go \N \Gd \N \sepnl
\Da \Go \N \Gd \N \sepnl


Ce qui donne ensuite, détaillé, le moudi conserve la structure

\vspace{1em}
\matra{\Da \Go \N \N} \sep \matra{\Go \N \Da \Go} \sepnl
\matra{\N \Gd \N \Da} \sep \matra{\Go \N \Gd \N} \sepnl

\matra{\Ta \K \N \N} \sep \matra{\K \N \Da \Go} \sepnl
\matra{\N \Gd \N \Da} \sep \matra{\Go \N \Gd \N} \sepnl

\var{2}

\Da \Go \N \N \Go \N \sep \Da \Go \N \N \Go \N \sepnl
\Da \Go \N \N \sepnl
+ 1/2 thème
% ****************************************************************************
% ***************************************************************** Références 
% ****************************************************************************
%\section{Références}

\end{document}
